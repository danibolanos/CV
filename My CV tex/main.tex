\documentclass{resume}

\begin{document}

\fontfamily{ppl}\selectfont

\noindent
\begin{tabularx}{\linewidth}{@{}m{0.76\textwidth} m{0.1\textwidth}@{}}
{
    \Large{Daniel Bolaños Martínez} \newline
    \vspace{-8mm}
    
    \small{
        \clink{
          \textbf{Email:}  \href{mailto:bolanosmartinezdaniel@gmail.com}{bolanosmartinezdaniel@gmail.com}
        \newline
        \textbf{LinkedIn:} \href{https://www.linkedin.com/in/danielbolanosm}{linkedin.com/in/danielbolanosm}
        \newline
        \textbf{GitHub:} \href{https://github.com/danibolanos}{github.com/danibolanos}
        }
    }
} & 
{
}
\end{tabularx}
\vspace{5mm}
\begin{center}
\begin{tabularx}{\linewidth}{@{}*{2}{X}@{}}
% left side %
{
    \csection{EDUCATION}{\small
        \begin{itemize}
            % item 1 %
            \item \frcontent{Bachelor Degree in Computer Science}{University of Granada, Spain.}{}{2015-2021}
            \item \frcontent{Bachelor Degree in Mathematics}{University of Granada, Spain.}{}{2015-2021}
        \end{itemize}
    }
    \csection{PROJECTS}{\small
        \begin{itemize}
            \item \frcontent{Ensuring Fairness in Machine Learning}{\textbf{Bachelor's Thesis.}}{Analysis of different fairness metrics and interpreting their results for a model implemented in Python using Aequitas API.}{\href{https://github.com/danibolanos/TFG-Ensuring_Fairness_in_ML.git}{github.com/TFG-Ensuring\_Fairness\_in\_ML}}
            \item \frcontent{Seam-Carving Algorithm}{\textbf{Project for Computer Vision subject.}}{Software to resize images while preserving the most relevant content using the OpenCV-Python library.}{\href{https://github.com/danibolanos/Computer_Vision_UGR/tree/master/Final\%20Project}{github.com/Seam-Carving\_Algorithm}}
            \item \frcontent{Data Warehouse PDI/Mondrian}{\textbf{Project for Multidimensional Analysis subject.}}{Development of a business intelligence project creating the ETL schema with Kettle and the OLAP component with Mondrian.}{\href{https://github.com/danibolanos/Practicas_SMD/tree/master/Final\%20Project}{github.com/Data-Warehouse\_Project}}
        \end{itemize}
    }
     \csection{COMPUTER SKILLS}{\small
        \begin{itemize}
            \item \textbf{Advanced Knowledge} \newline
            {\footnotesize Python, C++, \LaTeX{}, Git, Linux}
            \item \textbf{Intermediate Knowledge} \newline
            {\footnotesize Java, Keras, OpenCV, KNIME, PowerQuery}
            \item \textbf{Basic Knowledge} \newline
            {\footnotesize Ruby, C, SQL, Kettle PDI, R, HTML, CSS}
        \end{itemize}
    }
} 
% end left side %
& 
% right side %
{
    \csection{COURSES}{\small
        \begin{itemize}
            \item \frcontent{Data Warehouse for Business Intelligence}{Levelcom, 40 hours}{}{September 2020}
            \item \frcontent{ASP.NET Programming}{Levelcom, 40 hours}{}{November 2020}
            \item \frcontent{SPSS Programming}{EventEX, 25 hours}{}{May 2021}
        \end{itemize}
    }
    \csection{LANGUAGES}{\small
        \begin{itemize}
            \item \frcontent{Spanish}{}{}{Mothertongue}
            \vspace{-2mm}
            \item \frcontent{English}{Cambridge First Certificate in English (FCE).}{}{August 2015}
            \vspace{-2mm}
            \item \frcontent{French}{}{}{Basic Knowledge}
        \end{itemize}
    }
    \csection{EXTRA}{\small
        \begin{itemize}
            \item \frcontent{Interested in:}{Machine learning, Data Science, Business Intelligence, Deep Learning, Open Source, Software Development.}{}{}
            \vspace{-7mm}
            \item {\footnotesize Currently studying for the Cambridge Certificate in Advanced English (CAE).}
            \vspace{-1mm}
            \item {\footnotesize Intermediate knowledge of image and video processing: Photoshop \& Sony Vegas.}
            \vspace{-1mm}
            \item {\footnotesize Good teamwork and task organisation skills.}
            \vspace{-1mm}
            \item {\footnotesize Driving license and own vehicle.}
        \end{itemize}
    }
}
\end{tabularx}
\end{center}
\end{document}